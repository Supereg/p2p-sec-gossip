\documentclass[a4paper, 11pt]{article}
\usepackage[utf8]{inputenc}
\usepackage{cite}
\usepackage{amsmath,amssymb,amsfonts}
\usepackage{algorithmic}
\usepackage{graphicx}
\usepackage{textcomp}
\usepackage{xcolor}
\usepackage{hyperref}
\usepackage{multirow}
\usepackage{color}
\usepackage{listings}
\usepackage[left=25mm , right=25mm, top=25mm, bottom=25mm]{geometry}

\newcommand{\ilc}[1]{\texttt{#1}} % inline-code

\title{Initial Report}
\author{
    Andreas Bauer\\
    % Technical University of Munich\\
    andi.bauer@tum.de
  \and
    Chung Hwan Han\\
    hanc@in.tum.de
}
\date{May 2021}

\begin{document}

    \maketitle

    \section{Objective}\label{sec:objective}

    We, Andreas Bauer and Chung Hwan Han, form the team \textit{Gossip-10} in the Peer-to-Peer Systems and Security course.
    Within the project we implement the Gossip module of the \textit{VoidPhone} Voice over IP system.
    The system provides anonymity and understandability to voice communication through a peer-to-peer architecture.
    The Gossip module is used to spread information like availability information among communicating peers.
    Other modules from other teams might rely on the Gossip module to spread information required for their operations
    through the network.
    For this purpose, the project specification defines a socket-layer API for inter-module communication.
    The protocol spoken between Gossip instances is the core product of this project work.
    We follow the material and contents taught in the lecture -- considering best practices, common attacks and security measures --
    for the instantiation of our P2P protocol.

    \section{Tooling and Infrastructure}\label{sec:tooling-and-infrastructure} % TODO maybe rework wording of the headline
    % chosen programming language (+os?)
    % build systen and infrastructure: gradle
    % CI/CD => tests, linting etc
    % available libraries (netty for asny net I/O, bouncy castle crypto?, init4j,JUint 5)

    % licenscing % TODO in which section to put this?

    \section{Previous Experiences}\label{sec:previous-experiences}

    % programming experiences of team members `which is relevatn to this project' % TODO what means ``relevant'' here?

    % TODO timeline? + work distribution
\end{document}
