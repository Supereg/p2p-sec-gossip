\section{Requirements}\label{sec:requirements}

Within the work spent on the project, we derived several requirements set against the system.
This section provides an overview of what the Gossip is supposed to do and what we identify as essential
for our ongoing design decisions.

As roughly outlined above, Gossip is responsible for spreading information in the peer-to-peer network.
To do so, Gossip establishes connections with several other members of the network.
Each peer maintains a public-private key pair\footnote{A 4096-bit long RSA keypair.} -- referred to
as \textit{hostkey} -- which is used for peer identification and trust verification.
By specification, we assume that the public keys of the hostkeys are exchanged out-of-band
beforehand between all members.

Our developed protocol can currently be divided into two phases: \textit{handshake} and \textit{knowledge-spreading}.
For the handshake, we identify the following requirements:
\begin{itemize}
    \setlength\itemsep{0em}
    \item Verify the identity and authenticity of the remote peer.
    \item Establish a secure channel, providing confidentiality and integrity to the communication
    and preventing typical attacks like message replays.
    \item Ensure that as much meta-information as possible is protected (e.g., outside attackers can't easily
    trace who is communicating/connecting with whom except through information leaked by lower layers).
\end{itemize}

While knowledge-spreading is specified to be best-effort and doesn't need to make any guarantees,
we nonetheless want to deliver the best possible performance and protection against typical attacks.
Some attacks are already out-ruled by design by the specification (e.g., information is always validated
by the upper layer before continuing to spread information;
or knowledge is spread to the whole network and not to single identities).
Still, information may be lost if an Eclipse attack is staged and participating attacker peers drop all messages
of a particular data type.
Therefore, measures must be taken so that attackers don't have significant enough control over who is
initiating a connection to whom.
For routing itself, we must consider that we are maintaining an unstructured network and therefore need to
take measures that routing packets aren’t traveling forever in cycles.

Other modules interact with Gossip through the specification-defined TCP socket interface.
API consumers will register data types they are interested in to be notified about.
They may announce data into the network for those registered data types.
For incoming data, Gossip delivers it to all registered API clients.
Once each of those validated the data, information will be spread further into the network.
