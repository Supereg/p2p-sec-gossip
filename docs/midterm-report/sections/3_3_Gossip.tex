\subsection{Gossip Protocol}\label{subsec:gossip-protocol}

The \textbf{Gossip Protocol} component implements the protocol spoken between individual Gossip peers.
We rely on TCP to create a strongly connected network and rely on retransmissions and congestion control.

As outlined in the project specification, hostkeys are distributed out-of-band.
Therefore, we currently don't consider identity exchange (see \autoref{subsec:gossip-protocol-future-work}).
Known identities with their respective public keys and last known connection information are stored
on disk inside the \ttt{identities} folder controlled by the \textbf{IdentityStorage} component.
For testing purposes, we supply the \ttt{generateHostKey.sh} script to generate hostkeys for testing purposes.

Our protocol consists of two phases, the initial \textit{handshake} phase and the \textit{knowledge-spreading} phase
for established connections.
The \ttt{DISCONNECT} is the only packet valid in both phases and which might be sent at any time.
Its structure is depicted in \autoref{fig:gossip-packet-disconnect}.
It consists of a single \textit{reason} field describing the disconnect reason.
We currently maintain the following possible reasons: \ttt{NORMAL(0)}, \ttt{UNSUPPORTED(1)}, \ttt{AUTHENTICATION(64)},
\ttt{UNEXPECTED\_FAILURE(65)}, and \ttt{CANCELLED(66)}.

\begin{figure}[h!]
    \centering
    \begin{bytefield}{32}
        \bitheader{0,7-8,15-16,23-24,30-31} \\
        \bitbox{16}{size} & \bitbox{16}{\texttt{DISCONNECT}}\\
        \bitbox{8}{reason} & \bitbox{24}{reserved} \\
    \end{bytefield}
    \caption{GossipPacketDisconnect}
    \label{fig:gossip-packet-disconnect}
\end{figure}

The two protocol phases are described in detail in the following two subsections.

\subsubsection{Handshake}

For the handshake, we recall the requirements set in \autoref{sec:requirements}.
The primary purpose of the handshake is to create a secure channel for communication and verify
and authenticate identity claims of the peers mutually.

We rely on \textit{TLS 1.3}, serving us with a reliable, secure, and well-tested secure channel implementation.
Consequentially, we employ ECDHE (Elliptic Curve Diffie Hellman with an Ephemeral key) for the key exchange,
providing perfect forward secrecy for ongoing traffic.
For application-data, encryption we configure the AEAD cipher \textit{ChaCha20-Poly1305} supporting our confidentiality
and integrity security goals.

Our TLS layer is configured to do mutual authentication, meaning both the server and the client provide respective
certificates.
The root of trust for a peer's certificate chain is derived from its hostkey (a self-signed certificate containing
the public part of the hostkey).
This certificate acts as a Certificate Authority to an intermediate certificate used within the TLS authentication phase.
Within our \ttt{TrustManager} implementation, we verify the integrity of the certificate chain and ensure that
the root certificate is self-signed with the expected hostkey identity.\footnote{
To our best knowledge, we can't use an RSA-based certificate directly with TLS 1.3. Therefore, we employ this certificate chain.
Within the below-outlined handshake process, we validate that the remote peer is in possession of the hostkey's private key
    through a challenge-response procedure.
}
Certificate transmissions are encrypted with TLS 1.3.
Therefore, this doesn't leak the peer's identity at the connection establishment.
Each peer verifies that the TLS session was established with the expected hostkey ensuring that no man-in-the-middle
attack is in progress.

After completing the TLS handshake, we continue with the actual gossip handshake to verify possession of
the hostkey's private key.

\paragraph{Handshake Hello}

While the hostkey-based TLS certificate proves that the peer was at some point in possession of the hostkey,
we want to verify that the remote owns the hostkey right now.
Therefore, we employ a simple challenge-response procedure.

After the TLS handshake completes, the connection-initiating peer sends the
\ttt{HANDSHAKE HELLO} (see \autoref{fig:gossip-packet-handshake-hello}) packet to the server.
The packet specifies the used protocol \textit{version} (currently always \ttt{VERSION\_1(1)}).
Lastly, the packet contains a random 64-bit challenge with the request to be signed with the server's hostkey.

\begin{figure}[h!]
    \centering
    \begin{bytefield}{32}
        \bitheader{0,7-8,15-16,23-24,30-31} \\
        \bitbox{16}{size} & \bitbox{16}{\texttt{HANDSHAKE HELLO}}\\
        \bitbox{8}{version} & \bitbox{24}{reserved} \\
        \wordbox{2}{server challenge} \\
    \end{bytefield}
    \caption{GossipPacketHandshakeHello}
    \label{fig:gossip-packet-handshake-hello}
\end{figure}

\paragraph{Identity Verification 1}

The \ttt{IDENTITY VERIFICATION 1} packet is the response to the \ttt{HANDSHAKE HELLO}.
The packet structure is depicted in \autoref{fig:gossip-packet-handshake-identity-verification-1}.
The 512-byte long \textit{signature} field contains the server's response to the client-imposed challenge
(signs the hash of the concatenation of a common prefix, the imposed challenge, and the hostkey identity).
The \textit{client challenge} is a random 64-bit challenge with the request to be signed with the client's hostkey.

At this point, the client has successfully verified the server's identity.
In case of signature errors, the connection will be terminated after sending a corresponding \ttt{DISCONNECT} packet.

\begin{figure}[h!]
    \centering
    \begin{bytefield}{32}
        \bitheader{0,7-8,15-16,23-24,30-31} \\
        \bitbox{16}{size} & \bitbox{16}{\texttt{IDENTITY VERIFICATION 1}}\\
        \wordbox[tlr]{1}{signature} \\
        \skippedwords \\
        \wordbox[lr]{1}{} \\
        \wordbox{2}{client challenge} \\
    \end{bytefield}
    \caption{GossipPacketHandshakeIdentityVerification1}
    \label{fig:gossip-packet-handshake-identity-verification-1}
\end{figure}

\paragraph{Identity Verification 2}

The client sends the \ttt{IDENTITY VERIFICATION 2} packet in response to \ttt{IDENTITY VERIFICATION 1}.
The packet structure is depicted in \autoref{fig:gossip-packet-handshake-identity-verification-2}.
It contains the \textit{signature} (512 bytes) to the challenge imposed in the previous packet.

At this point, the server has successfully verified the client's identity.
In case of signature errors, the connection will be terminated after sending a corresponding \ttt{DISCONNECT} packet.
The server confirms the handshake by sending a \ttt{HANDSHAKE COMPLETE} packet to the client and switches to
the \textit{Knowledge-Spreading} phase.

\begin{figure}[h!]
    \centering
    \begin{bytefield}{32}
        \bitheader{0,7-8,15-16,23-24,30-31} \\
        \bitbox{16}{size} & \bitbox{16}{\texttt{IDENTITY VERIFICATION 2}}\\
        \wordbox[tlr]{1}{signature} \\
        \skippedwords \\
        \wordbox[lrb]{1}{} \\
    \end{bytefield}
    \caption{GossipPacketHandshakeIdentityVerification2}
    \label{fig:gossip-packet-handshake-identity-verification-2}
\end{figure}

\paragraph{Handshake Complete}

The \ttt{HANDSHAKE COMPLETE} (\autoref{fig:gossip-packet-handshake-complete}) is the explicit signal to the client
that the previous authentication steps were successful and that the client
should also switch to the \textit{Knowledge Spreading} phase.
% packet doesn't currently have any content, but in the future maybe!

\begin{figure}[h!]
    \centering
    \begin{bytefield}{32}
        \bitheader{0,7-8,15-16,23-24,30-31} \\
        \bitbox{16}{size} & \bitbox{16}{\texttt{HANDSHAKE COMPLETE}}\\
    \end{bytefield}
    \caption{GossipPacketHandshakeComplete}
    \label{fig:gossip-packet-handshake-complete}
\end{figure}

\subsubsection{Knowledge Spreading}

Once clients complete the \textit{Handshake} phase, they reside in the \textit{Knowledge Spreading} phase
and are active participants in the Gossip peer-to-peer network.

Once an API consumer subscribes to a data type (see \autoref{subsec:api-interface}), it may announce
data into the network.
To do so, we overtime propagate the \ttt{SPREAD KNOWLEDGE} packet to all connected peers (see \autoref{fig:gossip-spread-knowledge}).
For each announced data point, we generate a random 64-bit \textit{message id}.
In combination with the \textit{ttl} field (defined by the API consumer and decremented at every hop),
it provides means to avoid cycles in packet routing.

On arrival of a \ttt{SPREAD KNOWLEDGE}, we provide the data contents to all API connections subscribed to the
data type.
If there are none, we discard the packet.
Otherwise, we only forward the packet further into the network if all subscribed API connections report
validity of the packet contents.
This ensures that we don't propagate manipulated information into the network.

\begin{figure}[h!]
    \centering
    \begin{bytefield}{32}
        \bitheader{0,7-8,15-16,23-24,30-31} \\
        \bitbox{16}{size} & \bitbox{16}{\texttt{SPREAD KNOWLEDGE}}\\
        \wordbox[tlr]{1}{message id} \\
        \wordbox[lrb]{1}{} \\
        \bitbox{16}{ttl} & \bitbox{16}{data type} \\
        \wordbox[tlr]{1}{data} \\
        \skippedwords \\
        \wordbox[lrb]{1}{}
    \end{bytefield}
    \caption{GossipPacketSpreadKnowledge}
    \label{fig:gossip-spread-knowledge}
\end{figure}
