\section{Changes}\label{sec:objective}

With the final state of the project, we didn't identify any severe deviations from the project architecture
we anticipated in the initial report.

However, there were slight changes in the list of dependencies we used within the project.
Namely, we added the following new dependencies:

\begin{itemize}
    \setlength\itemsep{0em}
    \item \textbf{Guava}\footnote{https://guava.dev} is a commonly used utility library developed by Google, which
    we use in several locations.
    \item \textbf{Caffeine}\footnote{https://github.com/ben-manes/caffeine} is a high-performance, near-optimal
    caching library that builds upon the Guava Cache API\@.
    We use it within the \ttt{GossipModule} for data structures to enforce a time- or
    space-based expiry on its contents.
    \item \textbf{JCommander}\footnote{https://jcommander.org} is a library to parse command-line options in Java applications
    in a declarative way.
    (previously, in the midterm report, we used Apache Commons CLI before we switched to JCommander).
    \item \textbf{Gson}\footnote{https://github.com/google/gson} is a library from Google to
    make JSON Serialization and Deserialization easy.
    \item \textbf{commons-lang}\footnote{https://commons.apache.org/lang} is a library from
    the Apache Software Foundation that provides us with infrastructure to easily realize
    rate limiting.
    \item \textbf{commons-configuration2}\footnote{https://commons.apache.org/proper/commons-configuration/} is
    a library from the Apache Software Foundation that we use to parse the Windows INI file format.
    This replaces the \textit{init4j} library, our previous choice for INI file parsing.
    We replaced it, as it wasn't able to parse properties on the global section (e.g., the \textit{hostkey} option).
\end{itemize}
